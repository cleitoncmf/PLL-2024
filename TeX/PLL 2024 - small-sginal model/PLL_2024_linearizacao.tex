\documentclass[12pt,a4paper]{report}
\usepackage[utf8]{inputenc}
\usepackage{amsmath}
\usepackage{amsfonts}
\usepackage{amssymb}
\usepackage{graphicx}


\title{PLL 2024 - Linearização}

\begin{document}
\maketitle



Equações do PLL no domínio do tempo:



\begin{equation}
pll_\alpha = A \sin \theta
\label{eq:pll:alpha}
\end{equation}

\begin{equation}
pll_\beta = A \cos \theta
\label{eq:pll:beta}
\end{equation}

\begin{equation}
\varepsilon_\phi(t)=
\left(
v_\alpha - pll_\alpha
\right)
\cos \theta -
\left(
v_\beta - pll_\beta
\right)
\sin \theta
\label{eq:erro:phi}
\end{equation}

\begin{equation}
\varepsilon_A(t)=
\left(
v_\alpha - pll_\alpha
\right)
\sin \theta +
\left(
v_\beta - pll_\beta
\right)
\cos \theta
\label{eq:erro:amp}
\end{equation}




% ----------------------------
% ----------------------------
% ----------------------------
% ----------------------------
% ----------------------------
% ----------------------------
\section{Linearização Harmônica por Perturbação}

Para realizar a linearização por perturbação, devemos fazer as seguintes substituições:

\begin{equation}
\varepsilon_\phi =
\varepsilon_{\phi,ss} + 
\Delta \varepsilon_\phi
\end{equation} 

\begin{equation}
\varepsilon_A =
\varepsilon_{A,ss} + 
\Delta \varepsilon_A
\end{equation} 


\begin{equation}
v_\alpha =
v_{\alpha,ss} + 
\Delta v_\alpha
\end{equation} 

\begin{equation}
v_\beta =
v_{\beta,ss} + 
\Delta v_\beta
\end{equation} 


\begin{equation}
pll_\alpha =
pll_{\alpha,ss} + 
\Delta pll_\alpha
\end{equation} 

\begin{equation}
pll_\beta =
pll_{\beta,ss} + 
\Delta pll_\beta
\end{equation} 


\begin{equation}
\theta =
\theta_{ss} + \Delta \theta
\end{equation}

\begin{equation}
A = A_{ss} + \Delta A
\end{equation}


O processo de linearização das equações (\ref{eq:pll:alpha})-(\ref{eq:erro:amp}) será apresentado na sequência. Por ora os resultados ainda conterão termos senoidais/cossenoidais. 

Além disso, vou considerar:

\begin{equation}
\cos \Delta \theta \approx 1
\end{equation}

\begin{equation}
\sin \Delta \theta \approx \Delta \theta
\end{equation}

% ----------------------------
% ----------------------------
% ----------------------------
\subsection{Linearização da equação (\ref{eq:pll:alpha})}


\begin{equation}
pll_\alpha = A \sin \theta
\end{equation}



\begin{equation}
pll_{\alpha,ss} + \Delta pll_{\alpha} = 
\left(
A_{ss} + \Delta A
\right) 
\sin ( \theta_{ss} +  \Delta \theta )
\end{equation}


\begin{equation}
pll_{\alpha,ss} + \Delta pll_{\alpha} = 
\left(
A_{ss} + \Delta A
\right) 
\left(
\sin \theta_{ss} \cos \Delta \theta +
\cos \theta_{ss} \sin \Delta \theta
\right)
\end{equation}


\begin{equation}
pll_{\alpha,ss} + \Delta pll_{\alpha} = 
\left(
A_{ss} + \Delta A
\right) 
\left(
\sin \theta_{ss}  +
\cos \theta_{ss} \Delta \theta
\right)
\end{equation}


Coletando apenas os termos de primeira ordem:

\begin{equation}
\Delta pll_{\alpha} = 
A_{ss} \cos \theta_{ss} \Delta \theta +
\sin \theta_{ss} \Delta A
\label{eq:pll:alpha:Lin}
\end{equation}



% ----------------------------
% ----------------------------
% ----------------------------
\subsection{Linearização da equação (\ref{eq:pll:beta})}

\begin{equation}
pll_\beta = A \cos \theta
%\label{eq:pll:beta}
\end{equation}

\begin{equation}
pll_{\beta,ss} + \Delta pll_{\beta} = 
\left(
A_{ss} + \Delta A
\right)
\cos (\theta_{ss} + \Delta \theta )
\end{equation}

\begin{equation}
pll_{\beta,ss} + \Delta pll_{\beta} = 
\left(
A_{ss} + \Delta A
\right)
\left(
\cos \theta_{ss} \cos \Delta \theta -
\sin \theta_{ss} \sin \Delta \theta
\right)
\end{equation}



\begin{equation}
pll_{\beta,ss} + \Delta pll_{\beta} = 
\left(
A_{ss} + \Delta A
\right)
\left(
\cos \theta_{ss}  -
\sin \theta_{ss} \Delta \theta
\right)
\end{equation}


Coletando os termos de primeira ordem:

\begin{equation}
\Delta pll_{\beta} = 
- A_{ss} \sin \theta_{ss} \Delta \theta +
\cos \theta_{ss} \Delta A
\label{eq:pll:beta:Lin}
\end{equation}




% ----------------------------
% ----------------------------
% ----------------------------
\subsection{Linearização da equação (\ref{eq:erro:phi})}

\begin{equation}
\varepsilon_\phi(t)=
\left(
v_\alpha - pll_\alpha
\right)
\cos \theta -
\left(
v_\beta - pll_\beta
\right)
\sin \theta
%\label{eq:erro:phi}
\end{equation}


\begin{multline}
\varepsilon_{\phi,ss} + \Delta \varepsilon_\phi =
\left(
v_{\alpha,ss} + \Delta v_{\alpha} 
- pll_{\alpha,ss} - \Delta pll_\alpha
\right)
\cos (\theta_{ss} + \Delta \theta)\\ -
\left(
v_{\beta,ss} + \Delta v_\beta 
- pll_{\beta,ss} - \Delta pll_\beta
\right)
\sin (\theta_{ss} + \Delta \theta )
\end{multline}



\begin{multline}
\varepsilon_{\phi,ss} + \Delta \varepsilon_\phi =
\left(
v_{\alpha,ss} + \Delta v_{\alpha} 
- pll_{\alpha,ss} - \Delta pll_\alpha
\right)
\left(
\cos \theta_{ss} \cos \Delta \theta -
\sin \theta_{ss} \sin \Delta \theta
\right)\\ -
\left(
v_{\beta,ss} + \Delta v_\beta 
- pll_{\beta,ss} - \Delta pll_\beta
\right)
\left(
\sin \theta_{ss} \cos \Delta \theta +
\cos \theta_{ss} \sin \Delta \theta
\right)
\end{multline}








\begin{multline}
\varepsilon_{\phi,ss} + \Delta \varepsilon_\phi =
\left(
v_{\alpha,ss} + \Delta v_{\alpha} 
- pll_{\alpha,ss} - \Delta pll_\alpha
\right)
\left(
\cos \theta_{ss}  -
\sin \theta_{ss}  \Delta \theta
\right)\\ -
\left(
v_{\beta,ss} + \Delta v_\beta 
- pll_{\beta,ss} - \Delta pll_\beta
\right)
\left(
\sin \theta_{ss}  +
\cos \theta_{ss}  \Delta \theta
\right)
\end{multline}


Coletando os termos de primeira ordem:


\begin{multline}
\Delta \varepsilon_\phi = 
-v_{\alpha,ss} \sin \theta_{ss}  \Delta \theta
+\cos \theta_{ss} \Delta v_{\alpha} \\
+pll_{\alpha,ss} \sin \theta_{ss}  \Delta \theta
- \cos \theta_{ss}  \Delta pll_\alpha \\
- v_{\beta,ss} \cos \theta_{ss}  \Delta \theta
- \sin \theta_{ss} \Delta v_\beta \\
+ pll_{\beta,ss} \cos \theta_{ss}  \Delta \theta
+ \sin \theta_{ss}  \Delta pll_\beta
\end{multline}




\end{document}