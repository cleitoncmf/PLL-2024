\documentclass[12pt,a4paper]{report}
\usepackage[utf8]{inputenc}
\usepackage{amsmath}
\usepackage{amsfonts}
\usepackage{amssymb}
\usepackage{graphicx}
\usepackage{xcolor}


\title{PLL 2024 - Linearização}

\begin{document}
\maketitle



Equações do PLL no domínio do tempo:



\begin{equation}
pll_\alpha = A \sin \theta
\label{eq:pll:alpha}
\end{equation}

\begin{equation}
pll_\beta = A \cos \theta
\label{eq:pll:beta}
\end{equation}

\begin{equation}
\varepsilon_\phi(t)=
\left(
v_\alpha - pll_\alpha
\right)
\cos \theta -
\left(
v_\beta - pll_\beta
\right)
\sin \theta
\label{eq:erro:phi}
\end{equation}

\begin{equation}
\varepsilon_A(t)=
\left(
v_\alpha - pll_\alpha
\right)
\sin \theta +
\left(
v_\beta - pll_\beta
\right)
\cos \theta
\label{eq:erro:amp}
\end{equation}




% ----------------------------
% ----------------------------
% ----------------------------
% ----------------------------
% ----------------------------
% ----------------------------
\section{Linearização Harmônica por Perturbação}

Para realizar a linearização por perturbação, devemos fazer as seguintes substituições:

\begin{equation}
\varepsilon_\phi =
\varepsilon_{\phi,ss} + 
\Delta \varepsilon_\phi
\end{equation} 

\begin{equation}
\varepsilon_A =
\varepsilon_{A,ss} + 
\Delta \varepsilon_A
\end{equation} 


\begin{equation}
v_\alpha =
v_{\alpha,ss} + 
\Delta v_\alpha
\end{equation} 

\begin{equation}
v_\beta =
v_{\beta,ss} + 
\Delta v_\beta
\end{equation} 


\begin{equation}
pll_\alpha =
pll_{\alpha,ss} + 
\Delta pll_\alpha
\end{equation} 

\begin{equation}
pll_\beta =
pll_{\beta,ss} + 
\Delta pll_\beta
\end{equation} 


\begin{equation}
\theta =
\theta_{ss} + \Delta \theta
\end{equation}

\begin{equation}
A = A_{ss} + \Delta A
\end{equation}


O processo de linearização das equações (\ref{eq:pll:alpha})-(\ref{eq:erro:amp}) será apresentado na sequência. Por ora os resultados ainda conterão termos senoidais/cossenoidais. 

Além disso, vou considerar:

\begin{equation}
\cos \Delta \theta \approx 1
\end{equation}

\begin{equation}
\sin \Delta \theta \approx \Delta \theta
\end{equation}

% ----------------------------
% ----------------------------
% ----------------------------
\subsection{Linearização da equação (\ref{eq:pll:alpha})}


\begin{equation}
pll_\alpha = A \sin \theta
\end{equation}



\begin{equation}
pll_{\alpha,ss} + \Delta pll_{\alpha} = 
\left(
A_{ss} + \Delta A
\right) 
\sin ( \theta_{ss} +  \Delta \theta )
\end{equation}


\begin{equation}
pll_{\alpha,ss} + \Delta pll_{\alpha} = 
\left(
A_{ss} + \Delta A
\right) 
\left(
\sin \theta_{ss} \cos \Delta \theta +
\cos \theta_{ss} \sin \Delta \theta
\right)
\end{equation}


\begin{equation}
pll_{\alpha,ss} + \Delta pll_{\alpha} = 
\left(
A_{ss} + \Delta A
\right) 
\left(
\sin \theta_{ss}  +
\cos \theta_{ss} \Delta \theta
\right)
\end{equation}


Coletando apenas os termos de primeira ordem:

\begin{equation}
\Delta pll_{\alpha} = 
A_{ss} \cos \theta_{ss} \Delta \theta +
\sin \theta_{ss} \Delta A
\label{eq:pll:alpha:Lin}
\end{equation}



% ----------------------------
% ----------------------------
% ----------------------------
\subsection{Linearização da equação (\ref{eq:pll:beta})}

\begin{equation}
pll_\beta = A \cos \theta
%\label{eq:pll:beta}
\end{equation}

\begin{equation}
pll_{\beta,ss} + \Delta pll_{\beta} = 
\left(
A_{ss} + \Delta A
\right)
\cos (\theta_{ss} + \Delta \theta )
\end{equation}

\begin{equation}
pll_{\beta,ss} + \Delta pll_{\beta} = 
\left(
A_{ss} + \Delta A
\right)
\left(
\cos \theta_{ss} \cos \Delta \theta -
\sin \theta_{ss} \sin \Delta \theta
\right)
\end{equation}



\begin{equation}
pll_{\beta,ss} + \Delta pll_{\beta} = 
\left(
A_{ss} + \Delta A
\right)
\left(
\cos \theta_{ss}  -
\sin \theta_{ss} \Delta \theta
\right)
\end{equation}


Coletando os termos de primeira ordem:

\begin{equation}
\Delta pll_{\beta} = 
- A_{ss} \sin \theta_{ss} \Delta \theta +
\cos \theta_{ss} \Delta A
\label{eq:pll:beta:Lin}
\end{equation}




% ----------------------------
% ----------------------------
% ----------------------------
\subsection{Linearização da equação (\ref{eq:erro:phi})}

\begin{equation}
\varepsilon_\phi(t)=
\left(
v_\alpha - pll_\alpha
\right)
\cos \theta -
\left(
v_\beta - pll_\beta
\right)
\sin \theta
%\label{eq:erro:phi}
\end{equation}


\begin{multline}
\varepsilon_{\phi,ss} + \Delta \varepsilon_\phi =
\left(
v_{\alpha,ss} + \Delta v_{\alpha} 
- pll_{\alpha,ss} - \Delta pll_\alpha
\right)
\cos (\theta_{ss} + \Delta \theta)\\ -
\left(
v_{\beta,ss} + \Delta v_\beta 
- pll_{\beta,ss} - \Delta pll_\beta
\right)
\sin (\theta_{ss} + \Delta \theta )
\end{multline}



\begin{multline}
\varepsilon_{\phi,ss} + \Delta \varepsilon_\phi =
\left(
v_{\alpha,ss} + \Delta v_{\alpha} 
- pll_{\alpha,ss} - \Delta pll_\alpha
\right)
\left(
\cos \theta_{ss} \cos \Delta \theta -
\sin \theta_{ss} \sin \Delta \theta
\right)\\ -
\left(
v_{\beta,ss} + \Delta v_\beta 
- pll_{\beta,ss} - \Delta pll_\beta
\right)
\left(
\sin \theta_{ss} \cos \Delta \theta +
\cos \theta_{ss} \sin \Delta \theta
\right)
\end{multline}








\begin{multline}
\varepsilon_{\phi,ss} + \Delta \varepsilon_\phi =
\left(
v_{\alpha,ss} + \Delta v_{\alpha} 
- pll_{\alpha,ss} - \Delta pll_\alpha
\right)
\left(
\cos \theta_{ss}  -
\sin \theta_{ss}  \Delta \theta
\right)\\ -
\left(
v_{\beta,ss} + \Delta v_\beta 
- pll_{\beta,ss} - \Delta pll_\beta
\right)
\left(
\sin \theta_{ss}  +
\cos \theta_{ss}  \Delta \theta
\right)
\end{multline}


Coletando os termos de primeira ordem:


\begin{multline}
\Delta \varepsilon_\phi = 
-v_{\alpha,ss} \sin \theta_{ss}  \Delta \theta
+\cos \theta_{ss} \Delta v_{\alpha} \\
+pll_{\alpha,ss} \sin \theta_{ss}  \Delta \theta
- \cos \theta_{ss}  \Delta pll_\alpha \\
- v_{\beta,ss} \cos \theta_{ss}  \Delta \theta
- \sin \theta_{ss} \Delta v_\beta \\
+ pll_{\beta,ss} \cos \theta_{ss}  \Delta \theta
+ \sin \theta_{ss}  \Delta pll_\beta
\label{eq:erro:phi:Lin}
\end{multline}






% ----------------------------
% ----------------------------
% ----------------------------
\subsection{Linearização da equação (\ref{eq:erro:amp})}

\begin{equation}
\varepsilon_A(t)=
\left(
v_\alpha - pll_\alpha
\right)
\sin \theta +
\left(
v_\beta - pll_\beta
\right)
\cos \theta
%\label{eq:erro:amp}
\end{equation}


\begin{multline}
\varepsilon_{A,ss} + \Delta\varepsilon_A =
\left(
v_{\alpha,ss} + \Delta v_\alpha - 
pll_{\alpha,ss} + \Delta pll_\alpha
\right)
\sin (\theta_{ss} + \Delta)+\\
\left(
v_{\beta,ss} + \Delta v_\beta - 
pll_{\beta,ss} + \Delta pll_\beta
\right)
\cos (\theta_{ss} + \Delta \theta)
%\label{eq:erro:amp}
\end{multline}



\begin{multline}
\varepsilon_{A,ss} + \Delta\varepsilon_A =
\left(
v_{\alpha,ss} + \Delta v_\alpha - 
pll_{\alpha,ss} - \Delta pll_\alpha
\right)
\left(
\sin \theta_{ss} \cos \Delta \theta +
\cos \theta_{ss} \sin \Delta \theta
\right)+\\
\left(
v_{\beta,ss} + \Delta v_\beta - 
pll_{\beta,ss} - \Delta pll_\beta
\right)
\left(
\cos \theta_{ss} \cos \Delta \theta -
\sin \theta_{ss} \sin \Delta \theta
\right)
%\label{eq:erro:amp}
\end{multline}





\begin{multline}
\varepsilon_{A,ss} + \Delta\varepsilon_A =
\left(
v_{\alpha,ss} + \Delta v_\alpha - 
pll_{\alpha,ss} - \Delta pll_\alpha
\right)
\left(
\sin \theta_{ss} +
\cos \theta_{ss} \Delta \theta
\right)+\\
\left(
v_{\beta,ss} + \Delta v_\beta - 
pll_{\beta,ss} - \Delta pll_\beta
\right)
\left(
\cos \theta_{ss} -
\sin \theta_{ss}  \Delta \theta
\right)
%\label{eq:erro:amp}
\end{multline}



Coletando os termos de primeira ordem:


\begin{multline}
\Delta\varepsilon_A =
v_{\alpha,ss} \cos \theta_{ss} \Delta \theta 
+ \sin \theta_{ss}  \Delta v_\alpha \\
- pll_{\alpha,ss} \cos \theta_{ss} \Delta \theta
- \sin \theta_{ss} \Delta pll_\alpha \\
- v_{\beta,ss} \sin \theta_{ss}  \Delta \theta
+ \cos \theta_{ss} \Delta v_\beta \\
+ pll_{\beta,ss} \sin \theta_{ss}  \Delta \theta
- \cos \theta_{ss} \Delta pll_\beta
\label{eq:erro:amp:Lin}
\end{multline}












% ----------------------------
% ----------------------------
% ----------------------------
\subsection{Conjunto de equações linearizadas: versão inicial}

\begin{equation}
\Delta pll_{\alpha} = 
A_{ss} \cos \theta_{ss} \Delta \theta +
\sin \theta_{ss} \Delta A
%\label{eq:pll:alpha:Lin}
\end{equation}



\begin{equation}
\Delta pll_{\beta} = 
- A_{ss} \sin \theta_{ss} \Delta \theta +
\cos \theta_{ss} \Delta A
%\label{eq:pll:beta:Lin}
\end{equation}


\begin{multline}
\Delta \varepsilon_\phi = 
-v_{\alpha,ss} \sin \theta_{ss}  \Delta \theta
+\cos \theta_{ss} \Delta v_{\alpha} \\
+pll_{\alpha,ss} \sin \theta_{ss}  \Delta \theta
- \cos \theta_{ss}  \Delta pll_\alpha \\
- v_{\beta,ss} \cos \theta_{ss}  \Delta \theta
- \sin \theta_{ss} \Delta v_\beta \\
+ pll_{\beta,ss} \cos \theta_{ss}  \Delta \theta
+ \sin \theta_{ss}  \Delta pll_\beta
%\label{eq:erro:phi:Lin}
\end{multline}



\begin{multline}
\Delta\varepsilon_A =
v_{\alpha,ss} \cos \theta_{ss} \Delta \theta 
+ \sin \theta_{ss}  \Delta v_\alpha \\
- pll_{\alpha,ss} \cos \theta_{ss} \Delta \theta
- \sin \theta_{ss} \Delta pll_\alpha \\
- v_{\beta,ss} \sin \theta_{ss}  \Delta \theta
+ \cos \theta_{ss} \Delta v_\beta \\
+ pll_{\beta,ss} \sin \theta_{ss}  \Delta \theta
- \cos \theta_{ss} \Delta pll_\beta
%\label{eq:erro:amp:Lin}
\end{multline}






% ----------------------------
% ----------------------------
% ----------------------------
\subsection{Substituindo $v_{\alpha\beta,ss}$ e $pll_{\alpha\beta,ss}$}


N caso do PLL está sendo utilizado para extrair a componente de sequência negativa, podemos escrever:

\begin{equation}
v_{\alpha,ss} = A_{ss} \sin \theta_{ss}
\end{equation}

\begin{equation}
v_{\beta,ss} = A_{ss} \cos \theta_{ss}
\end{equation}


\begin{equation}
pll_{\alpha,ss} = A_{ss} \sin \theta_{ss}
\end{equation}

\begin{equation}
pll_{\beta,ss} = A_{ss} \cos \theta_{ss}
\end{equation}


Substituindo estes resultados em (\ref{eq:erro:phi:Lin}):

\begin{multline}
\Delta \varepsilon_\phi = 
- A_{ss} \sin \theta_{ss} \sin \theta_{ss}  \Delta \theta
+ \cos \theta_{ss} \Delta v_{\alpha} \\
+ A_{ss} \sin \theta_{ss} \sin \theta_{ss}  \Delta \theta
- \cos \theta_{ss}  \Delta pll_\alpha \\
- A_{ss} \cos \theta_{ss} \cos \theta_{ss}  \Delta \theta
- \sin \theta_{ss} \Delta v_\beta \\
+ A_{ss} \cos \theta_{ss} \cos \theta_{ss}  \Delta \theta
+ \sin \theta_{ss}  \Delta pll_\beta
%\label{eq:erro:phi:Lin}
\end{multline}

\begin{multline}
\Delta \varepsilon_\phi = 
- A_{ss} \sin^2 \theta_{ss}  \Delta \theta
+ \cos \theta_{ss} \Delta v_{\alpha} 
+ A_{ss} \sin^2 \theta_{ss}   \Delta \theta
- \cos \theta_{ss}  \Delta pll_\alpha \\
- A_{ss} \cos^2 \theta_{ss}  \Delta \theta
- \sin \theta_{ss} \Delta v_\beta 
+ A_{ss} \cos^2 \theta_{ss}   \Delta \theta
+ \sin \theta_{ss}  \Delta pll_\beta
%\label{eq:erro:phi:Lin}
\end{multline}



\begin{equation}
\Delta \varepsilon_\phi = 
 \cos \theta_{ss} \Delta v_{\alpha} 
- \cos \theta_{ss}  \Delta pll_\alpha 
- \sin \theta_{ss} \Delta v_\beta 
+ \sin \theta_{ss}  \Delta pll_\beta
\label{eq:erro:phi:Lin:2}
\end{equation}





Fazendo as substituições em (\ref{eq:erro:amp:Lin}):


\begin{multline}
\Delta\varepsilon_A =
A_{ss} \sin \theta_{ss} \cos \theta_{ss} \Delta \theta 
+ \sin \theta_{ss}  \Delta v_\alpha \\
- A_{ss} \sin \theta_{ss} \cos \theta_{ss} \Delta \theta
- \sin \theta_{ss} \Delta pll_\alpha \\
- A_{ss} \cos \theta_{ss} \sin \theta_{ss}  \Delta \theta
+ \cos \theta_{ss} \Delta v_\beta \\
+ A_{ss} \cos \theta_{ss} \sin \theta_{ss}  \Delta \theta
- \cos \theta_{ss} \Delta pll_\beta
%\label{eq:erro:amp:Lin}
\end{multline}


\begin{equation}
\Delta\varepsilon_A =
\sin \theta_{ss}  \Delta v_\alpha 
- \sin \theta_{ss} \Delta pll_\alpha 
+ \cos \theta_{ss} \Delta v_\beta 
- \cos \theta_{ss} \Delta pll_\beta
\label{eq:erro:amp:Lin:2}
\end{equation}










% ----------------------------
% ----------------------------
% ----------------------------
\subsection{Conjunto de equações linearizadas: versão simplificada}

\begin{equation}
\Delta pll_{\alpha} = 
A_{ss} \cos \theta_{ss} \Delta \theta +
\sin \theta_{ss} \Delta A
%\label{eq:pll:alpha:Lin}
\end{equation}



\begin{equation}
\Delta pll_{\beta} = 
- A_{ss} \sin \theta_{ss} \Delta \theta +
\cos \theta_{ss} \Delta A
%\label{eq:pll:beta:Lin}
\end{equation}




\begin{equation}
\Delta \varepsilon_\phi = 
\cos \theta_{ss} \Delta v_{\alpha} 
- \cos \theta_{ss}  \Delta pll_\alpha 
- \sin \theta_{ss} \Delta v_\beta 
+ \sin \theta_{ss}  \Delta pll_\beta
%\label{eq:erro:phi:Lin:2}
\end{equation}


\begin{equation}
\Delta\varepsilon_A =
\sin \theta_{ss}  \Delta v_\alpha 
- \sin \theta_{ss} \Delta pll_\alpha 
+ \cos \theta_{ss} \Delta v_\beta 
- \cos \theta_{ss} \Delta pll_\beta
%\label{eq:erro:amp:Lin:2}
\end{equation}




% ----------------------------
% ----------------------------
% ----------------------------
\subsection{Conjunto de equações linearizadas: usando Euler}


Antes de começar, temos que lembrar que $\theta_{ss} = \omega_{1}t$. Então:




\begin{equation}
\Delta pll_{\alpha} = 
A_{ss} 
\left(
\frac{e^{j\omega_1 t} + e^{-j\omega_1 t}}{2}
\right)
\Delta \theta +
\left(
\frac{e^{j\omega_1 t} - e^{-j\omega_1 t}}{2j}
\right)
\Delta A
\label{eq:pll:alpha:Lin:3}
\end{equation}



\begin{equation}
\Delta pll_{\beta} = 
- A_{ss} 
\left(
\frac{e^{j\omega_1 t} - e^{-j\omega_1 t}}{2j}
\right) 
\Delta \theta +
\left(
\frac{e^{j\omega_1 t} + e^{-j\omega_1 t}}{2}
\right)
\Delta A
\label{eq:pll:beta:Lin:3}
\end{equation}



\begin{multline}
\Delta \varepsilon_\phi = 
\left(
\frac{e^{j\omega_1 t} + e^{-j\omega_1 t}}{2}
\right)
\Delta v_{\alpha} 
- \left(
\frac{e^{j\omega_1 t} + e^{-j\omega_1 t}}{2}
\right)
\Delta pll_\alpha \\
- \left(
\frac{e^{j\omega_1 t} - e^{-j\omega_1 t}}{2j}
\right) 
\Delta v_\beta 
+ \left(
\frac{e^{j\omega_1 t} - e^{-j\omega_1 t}}{2j}
\right) 
\Delta pll_\beta
\label{eq:erro:phi:Lin:3}
\end{multline}




\begin{multline}
\Delta\varepsilon_A =
\left(
\frac{e^{j\omega_1 t} - e^{-j\omega_1 t}}{2j}
\right) 
\Delta v_\alpha 
- \left(
\frac{e^{j\omega_1 t} - e^{-j\omega_1 t}}{2j}
\right)
\Delta pll_\alpha \\
+ \left(
\frac{e^{j\omega_1 t} + e^{-j\omega_1 t}}{2}
\right) 
\Delta v_\beta 
- \left(
\frac{e^{j\omega_1 t} + e^{-j\omega_1 t}}{2}
\right) 
\Delta pll_\beta
\label{eq:erro:amp:Lin:3}
\end{multline}




% ----------------------------
% ----------------------------
% ----------------------------
\subsection{Conjunto de equações linearizadas: Representação usando vetores espaciais}

O objetivo desta seção é reescrever as equações da seção anterior com base nas seguintes definições:

\begin{equation}
\Delta \vec{v}_{\alpha\beta}^{~+}  = 
\Delta v_{\alpha} +
j \Delta v_{\beta}  
\end{equation}

\begin{equation}
\Delta \vec{v}_{\alpha\beta}^{~-}  = 
\Delta v_{\alpha} -
j \Delta v_{\beta}  
\end{equation}

\begin{equation}
\Delta \vec{pll}_{\alpha\beta}^{~+}  = 
\Delta pll_{\alpha} +
j \Delta pll_{\beta}  
\end{equation}


\begin{equation}
\Delta \vec{pll}_{\alpha\beta}^{~-}  = 
\Delta pll_{\alpha} -
j \Delta pll_{\beta}  
\end{equation}


Primeiramente, podemos combinar as equações (\ref{eq:pll:alpha:Lin:3}) e (\ref{eq:pll:beta:Lin:3}):

{
\color{blue}

\begin{multline}
\Delta \vec{pll}_{\alpha\beta} =
A_{ss} 
\left(
\frac{e^{j\omega_1 t} + e^{-j\omega_1 t}}{2}
\right)
\Delta \theta +
\left(
\frac{e^{j\omega_1 t} - e^{-j\omega_1 t}}{2j}
\right)
\Delta A \\ 
+ j\left[
- A_{ss} 
\left(
\frac{e^{j\omega_1 t} - e^{-j\omega_1 t}}{2j}
\right) 
\Delta \theta +
\left(
\frac{e^{j\omega_1 t} + e^{-j\omega_1 t}}{2}
\right)
\Delta A
\right]
\end{multline}



\begin{multline}
\Delta \vec{pll}_{\alpha\beta} =
A_{ss} 
\left(
\frac{e^{j\omega_1 t} + e^{-j\omega_1 t}}{2}
\right)
\Delta \theta +
j\left(
\frac{-e^{j\omega_1 t} + e^{-j\omega_1 t}}{2}
\right)
\Delta A \\ 
+ \left[
A_{ss} 
\left(
\frac{-e^{j\omega_1 t} + e^{-j\omega_1 t}}{2}
\right) 
\Delta \theta +
j \left(
\frac{e^{j\omega_1 t} + e^{-j\omega_1 t}}{2}
\right)
\Delta A
\right]
\end{multline}



}

\begin{equation}
\Delta \vec{pll}_{\alpha\beta} =
A_{ss} e^{-j\omega_1 t} \Delta \theta
+j e^{-j\omega_1 t} \Delta A
\end{equation}




Agora podemos processar a equação (\ref{eq:erro:phi:Lin:3}):

{
\color{blue}

\begin{multline}
\Delta \varepsilon_\phi = 
\left(
\frac{e^{j\omega_1 t} + e^{-j\omega_1 t}}{2}
\right)
\Delta v_{\alpha} 
- \left(
\frac{e^{j\omega_1 t} + e^{-j\omega_1 t}}{2}
\right)
\Delta pll_\alpha \\
- \left(
\frac{e^{j\omega_1 t} - e^{-j\omega_1 t}}{2j}
\right) 
\Delta v_\beta 
+ \left(
\frac{e^{j\omega_1 t} - e^{-j\omega_1 t}}{2j}
\right) 
\Delta pll_\beta
%\label{eq:erro:phi:Lin:3}
\end{multline}


\begin{multline}
\Delta \varepsilon_\phi = 
\left(
\frac{e^{j\omega_1 t} + e^{-j\omega_1 t}}{2}
\right)
\Delta v_{\alpha} 
- \left(
\frac{e^{j\omega_1 t} + e^{-j\omega_1 t}}{2}
\right)
\Delta pll_\alpha \\
+ j\left(
\frac{e^{j\omega_1 t} - e^{-j\omega_1 t}}{2}
\right) 
\Delta v_\beta 
- j\left(
\frac{e^{j\omega_1 t} - e^{-j\omega_1 t}}{2}
\right) 
\Delta pll_\beta
%\label{eq:erro:phi:Lin:3}
\end{multline}



\begin{multline}
\Delta \varepsilon_\phi =
\frac{1}{2} e^{j\omega_1 t}
( \Delta v_{\alpha}  + j\Delta v_{\beta} )
+\frac{1}{2} e^{-j\omega_1 t}
( \Delta v_{\alpha}  - j\Delta v_{\beta} )\\
-\frac{1}{2} e^{j\omega_1 t}
( \Delta pll_{\alpha}  + j\Delta pll_{\beta} )
-\frac{1}{2} e^{-j\omega_1 t}
( \Delta pll_{\alpha}  - j\Delta pll_{\beta} )
\end{multline}


}



\begin{equation}
\Delta \varepsilon_\phi =
\frac{1}{2} e^{j\omega_1 t}
\Delta \vec{v}_{\alpha\beta}^{~+}
+\frac{1}{2} e^{-j\omega_1 t}
\Delta \vec{v}_{\alpha\beta}^{~-}
-\frac{1}{2} e^{j\omega_1 t}
\Delta \vec{pll}_{\alpha\beta}^{~+}
-\frac{1}{2} e^{-j\omega_1 t}
\Delta \vec{pll}_{\alpha\beta}^{~-}
\end{equation}



Por último, temos que processar a equação (\ref{eq:erro:amp:Lin:3}):


{

\color{blue}

\begin{multline}
\Delta\varepsilon_A =
\left(
\frac{e^{j\omega_1 t} - e^{-j\omega_1 t}}{2j}
\right) 
\Delta v_\alpha 
- \left(
\frac{e^{j\omega_1 t} - e^{-j\omega_1 t}}{2j}
\right)
\Delta pll_\alpha \\
+ \left(
\frac{e^{j\omega_1 t} + e^{-j\omega_1 t}}{2}
\right) 
\Delta v_\beta 
- \left(
\frac{e^{j\omega_1 t} + e^{-j\omega_1 t}}{2}
\right) 
\Delta pll_\beta
%\label{eq:erro:amp:Lin:3}
\end{multline}


\begin{multline}
\Delta\varepsilon_A =
-j\left(
\frac{e^{j\omega_1 t} - e^{-j\omega_1 t}}{2}
\right) 
\Delta v_\alpha 
+j \left(
\frac{e^{j\omega_1 t} - e^{-j\omega_1 t}}{2}
\right)
\Delta pll_\alpha \\
+ \left(
\frac{e^{j\omega_1 t} + e^{-j\omega_1 t}}{2}
\right) 
\Delta v_\beta 
- \left(
\frac{e^{j\omega_1 t} + e^{-j\omega_1 t}}{2}
\right) 
\Delta pll_\beta
%\label{eq:erro:amp:Lin:3}
\end{multline}


\begin{multline}
\Delta\varepsilon_A =
\frac{1}{2}
e^{j\omega_1 t}
( -j \Delta v_{\alpha}  + \Delta v_{\beta} )
+\frac{1}{2}
e^{-j\omega_1 t}
( j \Delta v_{\alpha}  + \Delta v_{\beta} )\\
+\frac{1}{2}
e^{j\omega_1 t}
( j\Delta pll_{\alpha}  -\Delta pll_{\beta} )
+\frac{1}{2}
e^{-j\omega_1 t}
( -j\Delta pll_{\alpha}  -\Delta pll_{\beta} )
\end{multline}

\begin{multline}
\Delta\varepsilon_A =
-\frac{j}{2}
e^{j\omega_1 t}
( \Delta v_{\alpha}  +j \Delta v_{\beta} )
+\frac{j}{2}
e^{-j\omega_1 t}
( \Delta v_{\alpha}  -j \Delta v_{\beta} )\\
+\frac{j}{2}
e^{j\omega_1 t}
( \Delta pll_{\alpha}  +j\Delta pll_{\beta} )
-\frac{j}{2}
e^{-j\omega_1 t}
( \Delta pll_{\alpha}  -j \Delta pll_{\beta} )
\end{multline}


}


\begin{equation}
\Delta\varepsilon_A =
-\frac{j}{2}
e^{j\omega_1 t}
\Delta \vec{v}_{\alpha\beta}^{~+}
+\frac{j}{2}
e^{-j\omega_1 t}
\Delta \vec{v}_{\alpha\beta}^{~-}
+\frac{j}{2}
e^{j\omega_1 t}
\Delta \vec{pll}_{\alpha\beta}^{~+}
-\frac{j}{2}
e^{-j\omega_1 t}
\Delta \vec{pll}_{\alpha\beta}^{~-}
\end{equation}



\end{document}